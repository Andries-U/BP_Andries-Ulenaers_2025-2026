%==============================================================================
% Sjabloon onderzoeksvoorstel bachproef
%==============================================================================
% Gebaseerd op document class `hogent-article'
% zie <https://github.com/HoGentTIN/latex-hogent-article>

% Voor een voorstel in het Engels: voeg de documentclass-optie [english] toe.
% Let op: kan enkel na toestemming van de bachelorproefcoördinator!
\documentclass{hogent-article}

% Invoegen bibliografiebestand
\addbibresource{voorstel.bib}

% Informatie over de opleiding, het vak en soort opdracht
\studyprogramme{Professionele bachelor toegepaste informatica}
\course{Bachelorproef}
\assignmenttype{Onderzoeksvoorstel}
% Voor een voorstel in het Engels, haal de volgende 3 regels uit commentaar
% \studyprogramme{Bachelor of applied information technology}
% \course{Bachelor thesis}
% \assignmenttype{Research proposal}

\academicyear{2025-2026}

% Werktitel
\title{Maken visualisatietool voor het tonen van geschikte energieopwekkingssites ter ondersteuning van de einsteintelescoop}

% Studentnaam en emailadres invullen
\author{Andries Ulenaers}
\email{andries.ulenaers@student.hogent.be}

% TODO: Geef de co-promotor op
\supervisor[Co-promotor]{B. Vermang (UHasselt - PV Technologie Onderzoeksgroep , \href{mailto:bart.vermang@uhasselt.be}{bart.vermang@uhasselt.be})}


% Binnen welke specialisatierichting uit 3TI situeert dit onderzoek zich?
% Kies uit deze lijst:
%
% - Mobile \& Enterprise development
% - AI \& Data Engineering
% - Functional \& Business Analysis
% - System \& Network Administrator
% - Mainframe Expert
% - Als het onderzoek niet past binnen een van deze domeinen specifieer je deze
%   zelf
%
\specialisation{Mobile \& Enterprise development}
\keywords{Data visualisation, Solar energy, Gis}

\begin{document}

\begin{abstract}
  De Euregio Maas-Rijn is een kandidaat-locatie voor de Einstein Telescoop, een toekomstig Europees observatorium voor zwaartekrachtgolven. Om het biddingsproces van deze locatie te ondersteunen, is het essentieel om inzicht te verkrijgen in de mogelijkheden voor lokale duurzame energieopwekking rekening houdend met objectieve en subjectieve criteria. Deze bachelorproef richt zich op de ontwikkeling van een tool die visuele ondersteuning biedt voor het identificeren van geschikte locaties voor vershillende duurzame energieopwekkingsmethoden.
\end{abstract}

\tableofcontents

% De hoofdtekst van het voorstel zit in een apart bestand, zodat het makkelijk
% kan opgenomen worden in de bijlagen van de bachelorproef zelf.
\section{Inleiding}%
\label{sec:inleiding}

De Einstein Telescope (ET) wordt naar verwachting de meest gevoelige ondergrondse zwaartekrachtgolfdetector ter wereld. De Euregio Maas-\-Rijn, een grensregio die België, Nederland en Duitsland omvat, is één van de kandidaat-\-locaties voor dit grootschalig wetenschappelijk infrastructuurproject. Om deze locatie te laten weerhouden in het bidding process moet een uitgewerkt voorstel ingediend worden waarin onder andere een haalbare, duurzame energievoorziening wordt gedemonstreerd. Dit stelt uitzonderlijke eisen: niet alleen moet de energievoorziening betrouwbaar en stabiel zijn, maar door de extreme gevoeligheid van de meetapparatuur is ook trillingvrije energieopwekking en -\-levering essentieel. Daarnaast streeft het bidding-\-voorstel naar een maatschappelijk verantwoorde energiemix door samen te werken met lokale gemeenschappen en in te zetten op innovatieve hernieuwbare energiebronnen zoals Agro-\-Photovoltaics (AgriPV), warmte-\-krachtkoppeling (WKK) en diepe geothermie. Het voorbereiden van dit energievoorstel vereist een systematische analyse van de geografische mogelijkheden en beperkingen in de drielanden-\-regio.

Het plannen van een dergelijke energiemix wordt bemoeilijkt door meerdere factoren. Ten eerste is de locatie grensoverschrijdend, wat betekent dat geografische en infrastructurele data afkomstig is van verschillende regionale, nationale en internationale bronnen (Geopunt Vlaanderen, PDOK Nederland, GeoBasis-DE) met uiteenlopende formats, resoluties en toegankelijkheid. Ten tweede leggen Natura2000-\-gebieden en andere beschermde natuurzones strikte beperkingen op aan de plaatsing van energie-\-infrastructuur. Ten derde vereist de seismische gevoeligheid van de detector dat bepaalde energiebronnen (zoals windturbines in de nabijheid) uitgesloten worden of extra criteria moeten hanteren.

\textcolor{purple}{Deze bachelorproef gaat zich focussen op het ontwikkelen van een tool die GIS-datasets combineert met gebruikersgedefinieerde criteria en beperkingen ter ondersteuning van visuele site-selectie voor duurzame energievoorziening ter ondersteuning van een bidding voor de Einstein Telescope.}

% Doelgroep
Deze bachelorproef richt zich op de Einstein Telescope onderzoeksgroep en specifiek op de energie- en infrastructuurplanners die verantwoordelijk zijn voor het ontwerpen van een haalbare en duurzame energievoorziening voor de faciliteit. Secundair kunnen de resultaten ook relevant zijn voor andere grootschalige grensoverschrijdende infrastructuurprojecten die te maken hebben met complexe subjectieve criteria en beperkingen alsook multi-nationale data-integratie.

% Probleemstelling
De ET-onderzoeksgroep staat voor de uitdaging om te bepalen welke combinatie van duurzame energiebronnen technisch en geografisch haalbaar is voor hun specifieke locatie. Momenteel ontbreekt een geïntegreerde methode om systematisch geografische data uit verschillende regionale, nationale en internationale bronnen te combineren en te visualiseren. Het handmatig verzamelen, harmoniseren en analyseren van deze datasets is tijdsintensief en foutgevoelig. Bovendien is er behoefte aan een flexibel platform dat onderzoekers toelaat naast objectieve data gedreven criteria ook subjectieve criteria toe te voegen om de geschiktheid van een locatie te bepalen.

% Centrale onderzoeksvraag
De centrale onderzoeksvraag van dit onderzoek luidt: 

\textcolor{purple}{\textbf{Hoe kan een tool ontwikkeld worden die GIS-datasets combineert met gebruikersgedefinieerde criteria voor visuele site-selectie bij duurzame energievoorziening van grote infrastructuurprojecten?}}

Deze hoofdvraag wordt verder uitgesplitst in de volgende deelvragen:

\textit{Deelvragen met betrekking tot het probleemdomein:}
\textcolor{purple}{\begin{itemize}
    \item Wat zijn de specifieke technische eisen van de Einstein Telescope aan energievoorziening en welke beperkingen leggen deze op aan duurzame energiebronnen?
    \item Welke geografische, ecologische en infrastructurele beperkingen zijn van toepassing in de Euregio Maas-\-Rijn voor energie-\-infrastructuur?
    \item Welke GIS-\-databronnen zijn beschikbaar in België, Nederland en Duitsland, en hoe verschillen deze qua format, resolutie, actualiteit en toegankelijkheid?
    \item Welke bestaande tools of methoden worden momenteel gebruikt voor grensoverschrijdende GIS-\-analyse in energie-\-planning en wat zijn hun beperkingen?
    \item Welke datasets zijn nodig om de geschiktheid van de verschillende mogelijke energieoplossingen te bepalen?
\end{itemize}}

\textit{Deelvragen met betrekking tot het oplossingsdomein:}
\textcolor{purple}{\begin{itemize}
    \item Welke van de geidentificeerdebestaande tools geven de beste mogelijkheid om uit te breiden voor de specifieke doeleinden van de ET-onderzoeksgroep?
    \item Welke GIS-\-bibliotheken, frameworks en tools zijn geschikt voor het harmoniseren en analyseren van geografische datasets?
    \item Hoe kan een geschiktheids analyse geïmplementeerd worden die meerdere datasets combineert?
    \item Welke architectuur en technische implementatie laat toe dat onderzoekers op een gedocumenteerde, conflictvrije manier eigen analysescripts of plugins kunnen toevoegen?
    \item Hoe kunnen energie-\-yield schattingen voor verschillende bronnen (AgriPV, geothermie, WKK) geïntegreerd worden in de geografische analyse?
    \item Op welke wijze kan de output het meest effectief gevisualiseerd worden voor besluitvorming door niet-\-GIS-\-experts?
    \item Hoe kunnen gebruikersgedefinieerde criteria toegepast worden op de gecombineerde datasets?
    \item Hoe kunnen gebruikersgedefinieerde criteria op de meest optimale manier toegevoegd worden in de analyse?
\end{itemize}}

% Onderzoeksdoelstelling
Het concrete eindresultaat van dit onderzoek bestaat uit meerdere componenten:

\begin{enumerate}
    \item \textbf{Een proof-\-of-\-concept tool}: Een werkende applicatie die geografische datasets kan importeren, harmoniseren en analyseren op basis van configureerbare criteria. De tool moet uitbreidbaar zijn via een plugin-\-architectuur en voorzien zijn van uitgebreide technische documentatie voor toekomstige ontwikkelaars.
    
    \item \textbf{Een plugin voor de proof-\-of-\-concept tool}: Een uitgewerkte plugin die de proof-\-of-\-concept tool uitbreidt met de mogelijkheid om gebruikersgedefinieerde criteria te genereren en toe te voegen.
    
    \item \textbf{Een case study voor de Einstein Telescope}: Concrete toepassing van de tool op de ET-\-locatie(s), resulterend in:
    \begin{itemize}
        \item Een overzichtskaart van alle relevante constraints in de regio
        \item Suitability maps voor verschillende hernieuwbare energiebronnen
    \end{itemize}
\end{enumerate}

Dit onderzoek wordt als geslaagd beschouwd indien de ET-\-onderzoeksgroep de tool kan gebruiken om data-\-gedreven criteria met relevante maatschappelijke criteria te combineren om beslissingen te nemen over hun energiestrategie, en indien het platform aantoonbaar uitbreidbaar is voor toekomstige analyses door onderzoeksteams.

\section{Literatuurstudie}%
\label{sec:literatuurstudie}

Site-selectie voor hernieuwbare energievoorziening op basis van GIS-datasets is geen nieuw concept. De meeste studies baseren zich echter ofwel op objectieve criteria \autocite{Nassar2025}, ofwel op subjectieve criteria. De meeste studies beperken zich ook maar tot 1 geografische regio vanwege de complexiteit eenderzijds en anderszijds de zeldzaamheid van grensoverschrijdende locaties waar samenwerking vereist is. Er worden recent ook veel vooruitgangen geboekt op frameworks en werkmethoden voor multi-criteria analyses \autocite{Seran2025}. Deze zorgen dat we in deze praktische studie hiervan gebruik kunnen maken. Enkele van deze zullen verder onderzocht worden naar geschiktheid in deze casus. Hierbij is GIS-MCDA (Multi Criteria Decision Analysis) \autocite{Malczewski2015} de voornaamste kandidaat omwille van zijn flexibiliteit in het wegen van de verschillende criteria en bewezen toepassing in de hernieuwbare energievoorziening. Om deze selectiemethodes uit te werken zal er ook gekeken worden naar de meest gebruiksvriendelijke en open gis-tools gaande van QGis \autocite{QGIS2025} en verschillende Python-bibliotheken zoals GeoPandas \autocite{Geopandas2025}, PyQGis \autocite{PyQGIS2025} en GDAL \autocite{Rouault2025} om maar enkele voorbeelden te noemen.

% Hier beschrijf je de \emph{state-\-of-\-the-\-art} rondom je gekozen onderzoeksdomein, d.w.z.\ een inleidende, doorlopende tekst over het onderzoeksdomein van je bachelorproef. Je steunt daarbij heel sterk op de professionele \emph{vakliteratuur}, en niet zozeer op populariserende teksten voor een breed publiek. Wat is de huidige stand van zaken in dit domein, en wat zijn nog eventuele open vragen (die misschien de aanleiding waren tot je onderzoeksvraag!)?

% Je mag de titel van deze sectie ook aanpassen (literatuurstudie, stand van zaken, enz.). Zijn er al gelijkaardige onderzoeken gevoerd? Wat concluderen ze? Wat is het verschil met jouw onderzoek?

% Verwijs bij elke introductie van een term of bewering over het domein naar de vakliteratuur, bijvoorbeeld~\autocite{Hykes2013}! Denk zeker goed na welke werken je refereert en waarom.

% Draag zorg voor correcte literatuurverwijzingen! Een bronvermelding hoort thuis \emph{binnen} de zin waar je je op die bron baseert, dus niet er buiten! Maak meteen een verwijzing als je gebruik maakt van een bron. Doe dit dus \emph{niet} aan het einde van een lange paragraaf. Baseer nooit teveel aansluitende tekst op eenzelfde bron.

% Als je informatie over bronnen verzamelt in JabRef, zorg er dan voor dat alle nodige info aanwezig is om de bron terug te vinden (zoals uitvoerig besproken in de lessen Research Methods).

% Voor literatuurverwijzingen zijn er twee belangrijke commando's:
% \autocite{KEY} => (Auteur, jaartal) Gebruik dit als de naam van de auteur
%   geen onderdeel is van de zin.
% \textcite{KEY} => Auteur (jaartal)  Gebruik dit als de auteursnaam wel een
%   functie heeft in de zin (bv. ``Uit onderzoek door Doll & Hill (1954) bleek
%   ...'')

Je mag deze sectie nog verder onderverdelen in subsecties als dit de structuur van de tekst kan verduidelijken.

%---------- Methodologie ------------------------------------------------------
\section{Methodologie}%
\label{sec:methodologie}

Om de centrale onderzoeksvraag te beantwoorden, wordt het onderzoek uitgevoerd volgens een gestructureerd stappenplan met objectieve en meetbare onderzoekstechnieken. Onderstaand stappenplan beschrijft de verschillende fasen, geplande technieken, en gewenste deliverables.

\subsection{Stap 1: Vereistenanalyse}
\textbf{Doel:}
Identificeren van de vereisten en beperkingen voor het combineren van GIS-datasets en gebruikerscriteria, specifiek voor de Einstein Telescoop.

\textbf{Onderzoekstechniek:}
\begin{itemize}
    \item \textbf{Gestructureerde interviews} met stakeholders (ET-onderzoekers) om de benodigde datasets (bv. trillingsgevoeligheid, Natura 2000-gebieden) en criteria te bepalen per geplande energiebron.
    \item \textbf{Dataset-analyse} van bruikbaarheid van bestaande en eventueel aangeleverde datasets en bepalen welke datasets ontbreken.
\end{itemize}

\textbf{Deliverable:}
Een verslag met een overzicht van de benodigde datasets, criteria, en beperkingen.

\textbf{Tijdsduur:}
1 week.

\subsection{Stap 2: Vergelijkende Studie van Bestaande Tools}
\textbf{Doel:}
Selecteren van de meest geschikte open-source GIS-tool of library als basis voor verdere ontwikkeling.

\textbf{Onderzoekstechniek:}
\begin{itemize}
    \item \textbf{Vergelijkende literatuurstudie} van bestaande GIS-tools en libraries op basis van functionaliteit, gebruiksgemak, openheid en documentatie.
    \item \textbf{vergelijkende literatuurstudie} van opties om gebruikerscriteria om te vertalen zodat het geintegreerd kan worden in de site selectie.
    \item \textbf{Functionele analyse} van tools op ondersteuning voor GIS-formaten, visualisatieopties, en integratie van gebruikerscriteria.
\end{itemize}

\textbf{Deliverable:}
Een vergelijkende matrix met een gemotiveerde keuze voor de starttool.

\textbf{Tijdsduur:}
3 weken.

\subsection{Stap 3: Ontwikkeling en Uitbreiding van de Tool}
\textbf{Doel:}
Uitbreiden van de geselecteerde tool of framework om GIS-datasets te harmoniseren en gebruikerscriteria te integreren.

\textbf{Onderzoekstechniek:}
\begin{itemize}
    \item \textbf{Proof of Concept (PoC)}: Implementatie van functionaliteiten zoals:
    \begin{itemize}
        \item Inladen en harmoniseren van GIS-datasets (bv. conversie naar gemeenschappelijk formaat).
        \item Toevoegen van gebruikersgedefinieerde criteria (bv. trillingsgevoeligheid, wetgeving).
    \end{itemize}
    \item \textbf{Technische implementatie} met Python (GDAL, GeoPandas) en QGIS-plugins.
\end{itemize}

\textbf{Deliverable:}
Een werkende PoC met documentatie van de toegevoegde functionaliteiten.

\textbf{Tijdsduur:}
6 weken.

---

\subsection{Stap 4: Visualisatie en Validatie}
\textbf{Doel:}
Optimaliseren van de visualisatie van gecombineerde datasets en criteria, en valideren met echte data.

\textbf{Onderzoekstechniek:}
\begin{itemize}
    \item \textbf{Literatuurstudie} van bestaande visualisatietechnieken en bepalen welke technieken het beste passen voor de site selectie.
    \item \textbf{Experimenten} met visualisatietechnieken die zijn voortgekomen uit de literatuurstudie.
    \item \textbf{Validatie} met echte data uit het biddingsdossier van de Einstein Telescoop.
\end{itemize}

\textbf{Deliverable:}
Een rapport met de gekozen visualisatietechnieken en resultaten van de validatie.

\textbf{Tijdsduur:}
4 weken.

\subsection{Stap 5: Risicoanalyse}
\textbf{Doel:}
Identificeren en mitigeren van potentiële risico's (bv. datakwaliteit, technische beperkingen).

\textbf{Onderzoekstechniek:}
\begin{itemize}
    \item \textbf{Risicomatrix} om risico's in te schatten op impact en kans.
    \item \textbf{Mitigatiestrategieën} ontwikkelen voor geïdentificeerde risico's.
\end{itemize}

\textbf{Deliverable:}
Een risicoanalyseverslag met mitigatiestrategieën.

\textbf{Tijdsduur:}
1 week.

% Om de probleemstelling van de ET-onderzoeksgroep te bepalen, zal een interview worden gehouden met enkele researchers. Hieruit zal worden bepaald welke datasets gecombineerd moeten worden en op welke wijze dit moet gebeuren voor de verschillende duurzame energiebronnen die onderzocht worden. Dit interview zal ook extra beperkingen van het specifieke project verduidelijken.

% Er zal een vergelijkende studie worden gemaakt voor de verschillende bestaande tools die delen van de probleemstelling kunnen oplossen om zo de meeste geschikte starttool te bepalen.

% Ook moet er een vergelijkende studie worden gemaakt over de beste manier om gebruikersgedefinieerde criteria toe te voegen in de analyse.

% er zal moeten worden bepaald wat de beste manier is om deze datasets en criteria samen te visualiseren.



% Hier beschrijf je hoe je van plan bent het onderzoek te voeren. Welke onderzoekstechniek ga je toepassen om elk van je onderzoeksvragen te beantwoorden? Gebruik je hiervoor literatuurstudie, interviews met belanghebbenden (bv.~voor requirements-\-analyse), experimenten, simulaties, vergelijkende studie, risico-\-analyse, PoC, \ldots?

% Valt je onderwerp onder één van de typische soorten bachelorproeven die besproken zijn in de lessen Research Methods (bv.\ vergelijkende studie of risico-\-analyse)? Zorg er dan ook voor dat we duidelijk de verschillende stappen terug vinden die we verwachten in dit soort onderzoek!

% Vermijd onderzoekstechnieken die geen objectieve, meetbare resultaten kunnen opleveren. Enquêtes, bijvoorbeeld, zijn voor een bachelorproef informatica meestal \textbf{niet geschikt}. De antwoorden zijn eerder meningen dan feiten en in de praktijk blijkt het ook bijzonder moeilijk om voldoende respondenten te vinden. Studenten die een enquête willen voeren, hebben meestal ook geen goede definitie van de populatie, waardoor ook niet kan aangetoond worden dat eventuele resultaten representatief zijn.

% Uit dit onderdeel moet duidelijk naar voor komen dat je bachelorproef ook technisch voldoende diepgang zal bevatten. Het zou niet kloppen als een bachelorproef informatica ook door bv.\ een student marketing zou kunnen uitgevoerd worden.

% Je beschrijft ook al welke tools (hardware, software, diensten, \ldots) je denkt hiervoor te gebruiken of te ontwikkelen.

% Probeer ook een tijdschatting te maken. Hoe lang zal je met elke fase van je onderzoek bezig zijn en wat zijn de concrete \emph{deliverables} in elke fase?

%---------- Verwachte resultaten ----------------------------------------------
\section{Verwacht resultaat, conclusie}%
\label{sec:verwachte_resultaten}

Verwacht wordt dat er een tool ontwikkelt zal worden die zowel objectieve als subjectieve criteria gaat gebruiken om een heatmap te genereren. Deze heatmap visualiseert de meest optimale locaties voor verschillende duurzame energiebronnen. Aan de hand van deze tool zal de onderzoeksgroep een betere beeld kunnen krijgen van de meest optimale locaties. De tool zal de onderzoeksgroep in staat stellen om een overzicht te verkrijgen van de geschiktheid van locaties, gebaseerd op zowel technische als maatschappelijke factoren.

De tool zal uitgebreid gedocumenteerd zijn, zodat deze eenvoudig kan worden gebruikt door andere onderzoekers. Hierdoor kunnen toekomstige vraagstukken waar zowel objectieve als subjectieve criteria moeten onderzocht worden effectiever worden aangepakt.

\printbibliography[heading=bibintoc]

\end{document}