\section{Inleiding}%
\label{sec:inleiding}

% Kadering thema
De Einstein Telescope (ET) wordt naar verwachting de meest gevoelige ondergrondse zwaartekrachtgolf-detector ter wereld en zal gevestigd worden in de Euregio Maas-Rijn, een grensregio die België, Nederland en Duitsland omvat. Dit grootschalig wetenschappelijk infrastructuurproject stelt uitzonderlijke eisen aan zijn energievoorziening: niet alleen moet deze betrouwbaar en stabiel zijn, maar door de extreme gevoeligheid van de meetapparatuur is ook trillingvrije energieopwekking en -levering essentieel. Daarnaast streeft het project naar een duurzame, maatschappelijk verantwoorde energiemix door samen te werken met lokale gemeenschappen en in te zetten op innovatieve hernieuwbare energiebronnen zoals Agro-Photovoltaics (AgriPV), warmte-krachtkoppeling (WKK) en diepe geothermie.

De Einstein Telescope (ET) wordt naar verwachting de meest gevoelige ondergrondse zwaartekrachtgolf-detector ter wereld. De Euregio Maas-Rijn, een grensregio die België, Nederland en Duitsland omvat, is één van de kandidaat-locaties voor dit grootschalig wetenschappelijk infrastructuurproject. Om deze locatie te laten weerhouden in het bidding process moet een uitgewerkt voorstel ingediend worden waarin onder andere een haalbare, duurzame energievoorziening wordt gedemonstreerd. Dit stelt uitzonderlijke eisen: niet alleen moet de energievoorziening betrouwbaar en stabiel zijn, maar door de extreme gevoeligheid van de meetapparatuur is ook trillingvrije energieopwekking en -levering essentieel. Daarnaast streeft het bidding-voorstel naar een maatschappelijk verantwoorde energiemix door samen te werken met lokale gemeenschappen en in te zetten op innovatieve hernieuwbare energiebronnen zoals Agro-Photovoltaics (AgriPV), warmte-krachtkoppeling (WKK) en diepe geothermie. Het voorbereiden van dit energievoorstel vereist een systematische analyse van de geografische mogelijkheden en beperkingen in de drielanden-regio.

Het plannen van een dergelijke energiemix wordt bemoeilijkt door meerdere factoren. Ten eerste is de locatie grensoverschrijdend, wat betekent dat geografische en infrastructurele data afkomstig is van verschillende nationale bronnen (Geopunt Vlaanderen, PDOK Nederland, GeoBasis-DE) met uiteenlopende formats, resoluties en toegankelijkheid. Ten tweede leggen Natura2000-gebieden en andere beschermde natuurzones strikte beperkingen op aan de plaatsing van energie-infrastructuur. Ten derde vereist de seismische gevoeligheid van de detector dat bepaalde energiebronnen (zoals windturbines in de nabijheid) uitgesloten worden of extra criteria moeten hanteren.

% Doelgroep
Deze bachelorproef richt zich op de Einstein Telescope onderzoeksgroep en specifiek op de energie- en infrastructuurplanners die verantwoordelijk zijn voor het ontwerpen van een haalbare, duurzame energievoorziening voor de faciliteit. Secundair kunnen de resultaten ook relevant zijn voor andere grootschalige grensoverschrijdende infrastructuurprojecten die te maken hebben met complexe geografische constraints en multi-nationale data-integratie.

% Probleemstelling
De ET-onderzoeksgroep staat voor de uitdaging om te bepalen welke combinatie van hernieuwbare energiebronnen technisch en geografisch haalbaar is voor hun specifieke locatie. Momenteel ontbreekt een geïntegreerde methode om systematisch geografische data uit drie landen te combineren met technische constraints (seismische gevoeligheid, energiebehoefte) en ecologische beperkingen (Natura2000, beschermde gebieden). Het handmatig verzamelen, harmoniseren en analyseren van deze heterogene datasets is tijdsintensief en foutgevoelig. Bovendien is er behoefte aan een flexibel platform dat onderzoekers toelaat om eigen analyses toe te voegen via plugins of scripts, ondersteund door uitgebreide documentatie.

% Centrale onderzoeksvraag
De centrale onderzoeksvraag van dit onderzoek luidt: 

\textbf{Hoe kan een GIS-gebaseerde analyse- en visualisatietool de Einstein Telescope onderzoeksgroep ondersteunen bij het identificeren en evalueren van geschikte hernieuwbare energieopties, rekening houdend met grensoverschrijdende geografische data, seismische gevoeligheidseisen en ecologische beperkingen?}

Deze hoofdvraag wordt verder uitgesplitst in de volgende deelvragen:

\textit{Deelvragen met betrekking tot het probleemdomein:}
\begin{itemize}
    \item Wat zijn de specifieke technische eisen van de Einstein Telescope aan energievoorziening en welke beperkingen leggen deze op aan hernieuwbare energiebronnen?
    \item Welke geografische, ecologische en infrastructurele constraints zijn van toepassing in de Euregio Maas-Rijn voor energie-infrastructuur?
    \item Welke GIS-databronnen zijn beschikbaar in België, Nederland en Duitsland, en hoe verschillen deze qua format, resolutie, actualiteit en toegankelijkheid?
    \item Welke bestaande tools of methoden worden momenteel gebruikt voor grensoverschrijdende GIS-analyse in energie-planning en wat zijn hun beperkingen?
\end{itemize}

\textit{Deelvragen met betrekking tot het oplossingsdomein:}
\begin{itemize}
    \item Welke GIS-bibliotheken, frameworks en tools zijn geschikt voor het harmoniseren en analyseren van heterogene multi-nationale geografische datasets?
    \item Hoe kan een suitability analysis geïmplementeerd worden die meerdere constraint-lagen (ecologisch, seismisch, infrastructureel) combineert?
    \item Welke architectuur en technische implementatie laat toe dat onderzoekers op een gedocumenteerde, conflict-vrije manier eigen analysescripts of plugins kunnen toevoegen?
    \item Hoe kunnen energie-yield schattingen voor verschillende bronnen (AgriPV, geothermie, WKK) geïntegreerd worden in de geografische analyse?
    \item Op welke wijze kan de output het meest effectief gevisualiseerd worden voor besluitvorming door niet-GIS-experts?
\end{itemize}

% Onderzoeksdoelstelling
Het concrete eindresultaat van dit onderzoek bestaat uit drie componenten:

\begin{enumerate}
    \item \textbf{Een proof-of-concept tool}: Een werkende GIS-applicatie die multi-nationale geografische datasets kan importeren, harmoniseren en analyseren op basis van configureerbare criteria. De tool moet uitbreidbaar zijn via een plugin-architectuur en voorzien zijn van uitgebreide technische documentatie voor toekomstige ontwikkelaars.
    
    \item \textbf{Een methodologie-document}: Een gedetailleerde beschrijving van de workflow voor het harmoniseren van grensoverschrijdende GIS-data en het uitvoeren van multi-criteria suitability analyses, herbruikbaar voor soortgelijke projecten.
    
    \item \textbf{Een case study voor de Einstein Telescope}: Concrete toepassing van de tool op de ET-locatie(s), resulterend in:
    \begin{itemize}
        \item Een overzichtskaart van alle relevante constraints in de regio
        \item Suitability maps voor minimaal drie hernieuwbare energiebronnen (AgriPV, geothermie, WKK)
        \item Drie tot vijf uitgewerkte energiemix-scenario's met geografische plaatsing, yield-schattingen en een evaluatie op basis van technische haalbaarheid en ecologische impact
    \end{itemize}
\end{enumerate}

Dit onderzoek wordt als geslaagd beschouwd indien de ET-onderzoeksgroep de tool en methodologie kan gebruiken om data-gedreven beslissingen te nemen over hun energiestrategie, en indien het platform aantoonbaar uitbreidbaar is voor toekomstige analyses door het onderzoeksteam.

\section{Literatuurstudie}%
\label{sec:literatuurstudie}

Hier beschrijf je de \emph{state-of-the-art} rondom je gekozen onderzoeksdomein, d.w.z.\ een inleidende, doorlopende tekst over het onderzoeksdomein van je bachelorproef. Je steunt daarbij heel sterk op de professionele \emph{vakliteratuur}, en niet zozeer op populariserende teksten voor een breed publiek. Wat is de huidige stand van zaken in dit domein, en wat zijn nog eventuele open vragen (die misschien de aanleiding waren tot je onderzoeksvraag!)?

Je mag de titel van deze sectie ook aanpassen (literatuurstudie, stand van zaken, enz.). Zijn er al gelijkaardige onderzoeken gevoerd? Wat concluderen ze? Wat is het verschil met jouw onderzoek?

Verwijs bij elke introductie van een term of bewering over het domein naar de vakliteratuur, bijvoorbeeld~\autocite{Hykes2013}! Denk zeker goed na welke werken je refereert en waarom.

Draag zorg voor correcte literatuurverwijzingen! Een bronvermelding hoort thuis \emph{binnen} de zin waar je je op die bron baseert, dus niet er buiten! Maak meteen een verwijzing als je gebruik maakt van een bron. Doe dit dus \emph{niet} aan het einde van een lange paragraaf. Baseer nooit teveel aansluitende tekst op eenzelfde bron.

Als je informatie over bronnen verzamelt in JabRef, zorg er dan voor dat alle nodige info aanwezig is om de bron terug te vinden (zoals uitvoerig besproken in de lessen Research Methods).

% Voor literatuurverwijzingen zijn er twee belangrijke commando's:
% \autocite{KEY} => (Auteur, jaartal) Gebruik dit als de naam van de auteur
%   geen onderdeel is van de zin.
% \textcite{KEY} => Auteur (jaartal)  Gebruik dit als de auteursnaam wel een
%   functie heeft in de zin (bv. ``Uit onderzoek door Doll & Hill (1954) bleek
%   ...'')

Je mag deze sectie nog verder onderverdelen in subsecties als dit de structuur van de tekst kan verduidelijken.

%---------- Methodologie ------------------------------------------------------
\section{Methodologie}%
\label{sec:methodologie}

Hier beschrijf je hoe je van plan bent het onderzoek te voeren. Welke onderzoekstechniek ga je toepassen om elk van je onderzoeksvragen te beantwoorden? Gebruik je hiervoor literatuurstudie, interviews met belanghebbenden (bv.~voor requirements-analyse), experimenten, simulaties, vergelijkende studie, risico-analyse, PoC, \ldots?

Valt je onderwerp onder één van de typische soorten bachelorproeven die besproken zijn in de lessen Research Methods (bv.\ vergelijkende studie of risico-analyse)? Zorg er dan ook voor dat we duidelijk de verschillende stappen terug vinden die we verwachten in dit soort onderzoek!

Vermijd onderzoekstechnieken die geen objectieve, meetbare resultaten kunnen opleveren. Enquêtes, bijvoorbeeld, zijn voor een bachelorproef informatica meestal \textbf{niet geschikt}. De antwoorden zijn eerder meningen dan feiten en in de praktijk blijkt het ook bijzonder moeilijk om voldoende respondenten te vinden. Studenten die een enquête willen voeren, hebben meestal ook geen goede definitie van de populatie, waardoor ook niet kan aangetoond worden dat eventuele resultaten representatief zijn.

Uit dit onderdeel moet duidelijk naar voor komen dat je bachelorproef ook technisch voldoen\-de diepgang zal bevatten. Het zou niet kloppen als een bachelorproef informatica ook door bv.\ een student marketing zou kunnen uitgevoerd worden.

Je beschrijft ook al welke tools (hardware, software, diensten, \ldots) je denkt hiervoor te gebruiken of te ontwikkelen.

Probeer ook een tijdschatting te maken. Hoe lang zal je met elke fase van je onderzoek bezig zijn en wat zijn de concrete \emph{deliverables} in elke fase?

%---------- Verwachte resultaten ----------------------------------------------
\section{Verwacht resultaat, conclusie}%
\label{sec:verwachte_resultaten}

Hier beschrijf je welke resultaten je verwacht. Als je metingen en simulaties uitvoert, kan je hier al mock-ups maken van de grafieken samen met de verwachte conclusies. Benoem zeker al je assen en de onderdelen van de grafiek die je gaat gebruiken. Dit zorgt ervoor dat je concreet weet welk soort data je moet verzamelen en hoe je die moet meten.

Wat heeft de doelgroep van je onderzoek aan het resultaat? Op welke manier zorgt jouw bachelorproef voor een meerwaarde?

Hier beschrijf je wat je verwacht uit je onderzoek, met de motivatie waarom. Het is \textbf{niet} erg indien uit je onderzoek andere resultaten en conclusies vloeien dan dat je hier beschrijft: het is dan juist interessant om te onderzoeken waarom jouw hypothesen niet overeenkomen met de resultaten.

